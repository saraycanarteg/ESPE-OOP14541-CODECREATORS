% Zum Erstellen von lateinischen Fülltexten im Dokument
\usepackage{lipsum}

% Erforderlich zum Einfügen von Bildern
\usepackage{graphicx}

% Erlaubt die Verwendung von deutschen Datumsformaten
\usepackage[english]{datetime}

% Legt die Seitenränder des Dokuments fest
\usepackage[left=2cm, right=2cm, bottom=3cm, top=3cm]{geometry}

% Setzt den Zeilenabstand auf eineinhalb
\usepackage[onehalfspacing]{setspace}

% Ermöglicht die Verwendung von UTF-8-Kodierung für den Input
\usepackage[utf8]{inputenc}

% Paket für die Verwendung von Grafiken, mit Option für die Platzierungsoption [H]
\usepackage{float}

% Wird benötigt, um Grafiken an bestimmten Stellen im Text zu platzieren
\usepackage{enumerate}

% Erlaubt das Einbinden von URLs
\usepackage{url}

% Erlaubt das Einbetten von Hyperlinks im Dokument
\usepackage{hyperref}
\usepackage[nohyperlinks]{acronym}

% Erlaubt das Einbinden von Grafiken
\usepackage{graphicx}

% Erlaubt die Verwendung von Farben
\usepackage[]{xcolor}

% Erlaubt die Verwendung von Fußnoten
\usepackage{footnote}

% Erlaubt das Erstellen von Grafiken und Diagrammen
\usepackage{tikz}

% Erlaubt das Erstellen von Diagrammen mit TikZ
\usepackage{pgfplots}

% Erlaubt das Hinzufügen von Anmerkungen und To-Do-Notizen im Dokument
\usepackage{todonotes}

% Erlaubt das Einfügen von Blindtext zur Füllung von Platzhaltern im Dokument
\usepackage{lipsum}

% Ermöglicht die Verwendung der deutschen Sprache und stellt deutsche Sprachanpassungen bereit
\usepackage[english]{babel}

% Erlaubt die Verwendung von BibTeX zur Erstellung von Literaturverzeichnissen
\usepackage[backend=biber]{biblatex}
\addbibresource{Bibliography.bib} % Fügt die BibTeX-Datei für das Literaturverzeichnis hinzu

% Erlaubt die Verwendung von erweiterten mathematischen Funktionen und Symbolen
\usepackage{amsmath}

% Erlaubt das Umfließen von Bildern mit Text
\usepackage{wrapfig}

% Erlaubt das Schreiben von griechischen Buchstaben im Textmodus
\usepackage{textgreek}

% Erlaubt die Verwendung von Anführungszeichen mit Sprachanpassungen
\usepackage{csquotes}

% Erlaubt das Erstellen von Tabellen mit mehreren Reihen und Spalten
\usepackage{multirow}

% Setzt die PGFPlots-Kompatibilitätsversion auf 1.18
\pgfplotsset{compat=1.18}

% Fügt scrhack hinzu, um Warnungen zu vermeiden
\usepackage{scrhack} 

\setlength{\marginparwidth}{2cm}

%===================Kopf und Fußzeilen einstellen===============
\usepackage{scrlayer-scrpage} % vorhandene Kopf- und Fußzeilen löschen
\renewcommand*{\chapterpagestyle}{scrheadings} %Setzt den Seitenstil für Kapitelüberschriften
\clearpairofpagestyles

%===Kopfzeile===%
\ohead{\normalfont \today} % äußerer Kopf
\chead{\normalfont Code Creators} % zentraler Kopf
\ihead{\normalfont \rightmark}

%===Fußzeile===%
\ofoot{\normalfont Page~\pagemark} % äußerer Fuß
\cfoot{\normalfont OOP -14541}
\ifoot{\normalfont \myauthor} 

\KOMAoptions{headsepline=true,footsepline=true} % Strich unter der Kopfzeile

%============= Kein massiver Abstand zwischen Text und Kopfzeile =============
\renewcommand*{\chapterheadstartvskip}{\vspace*{-.4cm}}
\renewcommand*{\chapterheadendvskip}{\vspace{.5cm}}

\setlength{\parindent}{0pt} % no indent at paragraph start

% ============= Einstellung für die Größen der verschiedenen Überschriften =============
\setkomafont{chapter}{\LARGE}
\setkomafont{section}{\Large}
\setkomafont{subsection}{\large}
\setkomafont{subsubsection}{\normalsize}
\setkomafont{paragraph}{\normalsize}
\setkomafont{subparagraph}{\small}