% Created 2020-04-24 vie 23:03
% Intended LaTeX compiler: pdflatex
\documentclass[12pt,a4paper, twosite]{article}
\usepackage[utf8]{inputenc}
\usepackage[T1]{fontenc}
\usepackage{graphicx}
\usepackage{grffile}
\usepackage{longtable}
\usepackage{wrapfig}
\usepackage{rotating}
\usepackage[normalem]{ulem}
\usepackage{amsmath}
\usepackage{textcomp}
\usepackage{amssymb}
\usepackage{capt-of}
\usepackage{hyperref}
\usepackage[left=2.00cm, right=2.50cm, top=2.50cm, bottom=2.00cm]{geometry}
\usepackage{fancyhdr}
\fancyhead[RO,LE]{\thepage}
\fancyhead[LO]{\emph{\uppercase{\leftmark}}}
\fancyfoot{}
\renewcommand{\headrulewidth}{1.0pt}
\pagestyle{fancy}
\date{}
\title{CODE CREATORS}
\hypersetup{
 pdfauthor={},
 pdftitle={CODE CREATORS},
 pdfkeywords={},
 pdfsubject={},
 pdfcreator={Emacs 26.2 (Org mode 9.1.9)}, 
 pdflang={English}}
\begin{document}

\maketitle
\tableofcontents

\newpage

\section{Introduction}
\label{sec:org60390fa}

The purpose of this document is to define the software requirements for a cyber that performs tax and IVA declaration processes. The purpose of this system is to provide an efficient and secure tool to the cyber clients where they will be registered and their tax returns will be done in a simple and fast way.

The software will allow them to update their personal and fiscal data, as well as to present personalized notifications about tax payment deadlines and upcoming due dates. In addition, it will have a specific section for the management of prices established for the collection of the activities to be carried out, which will allow clients to have a clear and transparent view of the costs associated with each service.

With this system, the cyber will be able to offer its customers a more personalized and efficient experience, reducing waiting times and errors in tax reporting. In addition, the system will enable the cyber company to comply with the legal and regulatory requirements established by the tax authority, ensuring the security and reliability of customer information.

In summary, the main objective of the software to be developed is to provide an efficient and secure tool for tax and IVA returns, meeting the client's requirements and offering a personalized and transparent user experience.



\subsection{Purpose}
\label{sec:org434c3ef}

The purpose of the software project is to create a program to handle tax returns more efficiently. The program will provide an organized and secure tool for the owner of this business to be able to:
\begin{itemize}
\item \textbf{Register and update customers' personal and tax data:}  The main user will be able to add, update, search and delete tax payers with their own information (RUC, password, email, and others).

\item \textbf{Receive personalized notifications:} Provide personalized notifications about tax payment deadlines and upcoming due dates, so that Cyber Planeta plans and manages the tax processes to be carried out effectively by a classfication agenda and calendar.

\item \textbf{Calculate fees for services rendered:} 
The program will help the tax preparer calculate their fees for services rendered. This will include factors such as the type of return, the number of invoices, and any additional costs.

\end{itemize}


\subsection{System Scope}
\label{sec:org12e44a1}

 \textbf{Definition }
 \\“Cyber Planeta System” will be the name of the upcoming tax and VAT filing manager system. \\

\textbf{Functionalities }
\\The software empowers Cyber Planeta to streamline tax and VAT filing processes for their clients. The key functionalities will be:
\begin{itemize}
    \item Manage Tax Payer's data (edit, delete, add).
    \item Send automated datelines notifications.
    \item Calculate fees for services rendered.
    \item Display search results according to the ninth RUC digit.
    \item Show deadlines in calendar.
\end{itemize}

\textbf{Limitations}
\\
Cyber Planeta System will not perform the following functions:
\begin{itemize}
    \item It will not be an accounting or financial management system.
     \item It will not allow electronic payments
     \item It will not be a general tax information query system.
\end{itemize}

\textbf{Benefits}
\\
Cyber Planeta System is expected to provide the following benefits:
\begin{itemize}
    \item Increase efficiency and speed in tax and VAT filing.
    \item Improve accuracy and consistency of tax information.
    \item Increase security and reliability in the management of tax information.
    \item Improve data management and search.    
\end{itemize}


\textbf{Objectives and Goals}

The objectives and goals of Cyber Planeta System are:
\begin{itemize}
    \item Reduce tax and IVA filing time by 50 \%.
    \item Increase customer satisfaction by 80 \%.
    \item Reduce tax and IVA filing errors by 90 \%.    
\end{itemize}

\subsection{Definitions, Acronyms and Abbreviations}
\label{sec:orgb158e36}
\begin{itemize}
    \item \textbf{registerTaxPayer:} the process of registering a taxpayer through the program.
    \item \textbf{searchTaxPayer:} the process of searching a taxpayer through the program.
    \item \textbf{deleteTaxPayer:} the process of delete a taxpayer through the program.
    \item \textbf{classificateTaxPayer:} the taxpayer will be classified through the program.
    \item \textbf{calculateTaxProcessCost:} the cost to be charged to the taxpayer was calculated.
    \item \textbf{price:} Calculate the tax owed by a taxpayer.
    \item \textbf{typeofProcess:} process to be carried out for the taxpayer.
    \item \textbf{numberOfDocumentation:} where the taxpayer can be identified.
    \item \textbf{addDateCalendar:} the day when the declaration must be filed is added.
    \item \textbf{id:} taxpayer identification number or also known as NUI.
    \item \textbf{email:} email of taxpayer.
    \item \textbf{name:} taxpayer´s as identification.
    \item \textbf{password:} t is a secret word or phrase that is used to verify the identity of a user and grant them access to a resource, such as a computer system.
    \item \textbf{accountingDocumentation:} here the taxpayer's invoices and withholdings will be added.
    \item \textbf{deliveryDate:}deadline for filing the tax return.
    \item \textbf{startDate:} start date for filing the tax return.
    \item \textbf{sendNotification:} notifications will be sent days before the deadline so that processes do not accumulate.
    \item \textbf{numberOfFile:} "taxpayer's file number.
    \item \textbf{saveScannedFile:} as a backup for the business owner and also the customer.
    \item \textbf{processid:} the process to be carried out will be registered according to your identification.
    \item \textbf{processName:} each process will have its own name.
    \item \textbf{taxRate:} tax value
\end{itemize}


\subsection{References}
\label{sec:org62711e0}

Cyber Planeta System is based on the following higher level documents:
\begin{itemize}
    \item Global system requirements specification document.
    \item Cyber Information Security Policy
    \item Current tax rules and regulations
\end{itemize}


\subsection{Document Overview}
\label{sec:orgdaca22c}
This document is structured into the following sections:

\begin{itemize}
  \item \textbf{Introduction:} Presents the purpose, scope, definitions, references, and overview of the document.
  \item \textbf{General Description:} Describes the general characteristics of the "Cyber Planeta System".
  \item \textbf{Functional Requirements:} Details the specific functionalities and behaviors of the system.
  \item \textbf{Non-Functional Requirements:} Defines the non-functional characteristics of the system, such as performance, security, and usability.
  \item \textbf{Glossary:} Provides detailed definitions of technical terms used in the document.
  \item \textbf{Annexes:} Includes additional information, such as data flow diagrams and use cases.
\end{itemize}


\section{General Description}
\label{sec:orgc1c4017}

CyberPlanet is a comprehensive tax processes management system designed to streamline operations, enhance profitability, and deliver exceptional customer service. Its user-friendly interface and robust features cater to the needs of Guillermo Cañarte (Cyber Planeta owner).


\subsection{Product Perspective}
\label{sec:org24980a8}

The Cyber Planeta System is an independent product that does not require integration with other systems. It is a self-contained software solution that provides a comprehensive suite of features for managing tax processes data.


\subsection{Product Functions}
\label{sec:orgaf51da6}

Cyber Planeta streamlines tax payer's management and will focus in functionalities as: \begin{itemize}
\item \textbf{Tax Payer Management:}
\begin{itemize}
\item Add, edit, and delete tax payer profiles.
\item Attach individual documentation (e.g., scanned sales notes) to tax payer profiles.
\item Search by 9th RUC digit for easy identification of tax declaration day.
\end{itemize}
\item \textbf{Tax Declaration Management:}
\begin{itemize}
\item Display tax declarations in a convenient calendar format.
\item Organize tax services with a customizable schedule for process deliveries.
\end{itemize}
\item \textbf{Cost Calculation:}
\begin{itemize}
\item Empower the cyber owner to calculate the cost of each process.
\item Determine the base fee based on the process type.
\item Factor in additional costs based on process duration, documentation volume, and other relevant factors.
\end{itemize}
\end{itemize}


\subsection{User characteristics}
\label{sec:orga40b0ee}

The user of the product keeps the accounting of individuals and some companies, both those that are obliged to keep accounting and those that are not. He/she has access to the SRI's electronic information and must manage the access codes and the RUC. The user has basic knowledge of tax processes and the handling of both electronic information and physical documentation, such as sales notes. The user also has knowledge about the management for the dates of delivery according to the last number of the RUC. 


\subsection{Restrictions}
\label{sec:org5ca5790}

These are some limitations for the development of the program

\begin{itemize}
\item No data connection to SRI.

\item Do not store redundant information.

\item Multi-platform system.

\item Do not store information indefinitely.

\item It does not have to be a complex system.


\end{itemize}


\subsection{Assumptions and dependencies}
\label{sec:org0ae23fe}

The system assumes that the user will have all the taxpayer's data such as: id, email, name, password, and if he/she delivers physical documents, if any of this data is not obtained, the system will fail, as well as the date. Also the system assumes that the user will always have a computer and a database always available for information storage.


\subsection{Future requirements}
\label{sec:org33cfcdb}

A future requirement is for the system to send notifications about tax processes to the cell phone, in order to have a reminder and greater control over the process of each taxpayer.   


\section{Requisitos específicos}
\label{sec:org40573d1}

Esta sección contiene los requisitos a un nivel de detalle suficiente
como para permitir a los diseñadores diseñar un sistema que
satisfaga estos requisitos, y demuestren si el sistema satisface, o
no, los requisitos. Todo requisito aquí especificado describirá
comportamientos externos del sistema, perceptibles por parte de los
usuarios, operadores y otros sistemas. Esta es la sección más larga
e importante de la ERS. Deberán aplicarse los siguientes principios:

\begin{itemize}
\item El documento debería ser perfectamente legible por personas de muy
distintas formaciones e intereses.

\item Deberán referenciarse aquellos documentos relevantes que poseen
alguna influencia sobre los requisitos.

\item Todo requisito deberá ser unívocamente identificable mediante algún
código o sistema de numeración adecuado.

\item Lo ideal, aunque en la práctica no siempre realizable, es que los
requisitos posean las siguientes características: 

\begin{itemize}
\item \textbf{Corrección:} La ERS es correcta si y sólo si todo requisito que
figura aquí(y que será implementado en el sistema) refleja alguna
necesidad real. La corrección de la ERS implica que el sistema
implementado será el deseado.

\item \textbf{No ambiguos:} Cada requisito tiene una sola interpretación. Para
eliminar la ambigüedad inherente a los requisitos expresados en
lenguaje natural, se deberán utilizar gráficos o notaciones
formales. En el caso de utilizar términos que, habitualmente,
poseen más de una interpretación, se definirán con precisión en
glosario.

\item \textbf{Completos:} Todos los requisitos relevantes han sido incluidos en
la ERS. Conviene incluir todas las posibles respuestas del sistema
a los datos de entrada, tanto validos como no válidos.

\item \textbf{Consistentes:} Los requisitos no pueden ser contradictorios. Un
conjunto de requisitos contradictorios no es implementable.

\item \textbf{Clasificados:} Normalmente, no todos los requisitos son igual de
importantes. Los requisitos pueden clasificarse por importancia
(esenciales, condicionales u opcionales) o por estabilidad (cambios
que se espera que afecten al requisito). Esto sirve, ante todo,
para no emplear excesivos recursos en implementar requisitos no
esenciales.

\item \textbf{Verificables:} La ERS es verificalble si y sólo si todos sus
requisitos son verificables. Un requisito es verificable
(testeable) si existe un proceso finito y no costoso para
demostrar que el sistema cumple con el requisito. Un requisito
ambiguo no es, en general, verificable. Expresiones como a veces,
bien, adecuado, etc introducen ambigüedad en los
requisitos. Requisitos como "en caso de accidente la nube tóxica
no se extenderá más allá de 25km" no es verificable por el alto
costo que conlleva.

\item \textbf{Modificables:} La ERS es modificable si y sólo si se encuentra
estructurada de forma que los cambios a los requisitos puedan
realizarse de forma fácil, completa y consistente. La utilización
de herramientas automáticas de gestión de requisito (por ejemplo
RequisitePro o Doors) facilitan enormemente esta tarea.

\item \textbf{Trazables:} La ERS es trazable si se conoce el origen de cada
requisito y facilita la referencia de cada requisito a los
componentes y de la implementación. La trazabilidad hacia atrás
indica el origen (documento, persona, etc) de cada requisito. La
trazabilidad hacia delante de un requisito R indica qué
componentes del sistema son los que realizan el registro R.
\end{itemize}
\end{itemize}


\subsection{Interfaces externas}
\label{sec:orgfd5391f}

Se describirán los requisitos que afecten a la interfaz de usuario,
interfaz con otros sistemas (hardware y software) e interfaces de comunicaciones.


\subsection{Funciones}
\label{sec:org307bb59}

Esta subsección (quizás la más larga del documento) deberá
especificar todas aquellas acciones (funciones) que deberá llevar a
cabo el software. Normalmente (aunque no siempre) son aquellas
acciones expresables como "el sistema deberá \ldots{}" Si se considera
necesario, podrán utilizarse notaciones gráficas y tablas, pero
siempre supeditadas al lenguaje natural, y no al revés.

Es importante tener en cuenta que, en 1983, el estándar de IEEE 830
establecía que las funciones deberían expresarse como una jerarquía
funcional (en paralelo con los DFDs propuestas por el análisis
estructurado). Pero el estándar de IEEE 830, en sus últimas
versiones, ya permite organizar esta subsección de múltiples formas,
y sugiere, entre otras, las siguientes:


\begin{itemize}
\item Por tipos de usuarios: 
    Distintos usuarios poseen distintos requisitos. Para cada clase de
usuario que exista en la organización, se especificarán los
requisitos funcionales que le afecten o tengan mayor relación con
sus tareas.
\end{itemize}


\begin{itemize}
\item Por objetos:
   Los objetos son identidades del mundo real que serán reflejadas en
el sistema. Para cada objeto, se detallarán sus atributos y sus
funciones. Los objetos pueden agruparse en clases. Esta organización
de la ERS no quiere decir que el diseño del sistema siga el
paradigma de Orientación a Objetos.
\end{itemize}


\begin{itemize}
\item Por estímulos: 
  Se especificarán los posibles estímulos que recibe el sistema y las
funciones relacionadas con dicho estímulo.
\end{itemize}


\begin{itemize}
\item Por jerarquía funcional: 
   Si ninguna de las anteriores alternativas resulta de ayuda, la
funcionalidad del sistema se especificará como una jerarquía de
funciones que comparten entradas, salidas o datos internos. Se
detallarán las funciones (entrada, proceso, salida) y las
subfunciones del sistema. Esto no implica que el diseño del sistema
deba realizarse según el paradigma de diseño estructurado.
\end{itemize}


Para organizar esta subsección de la ERS se elegirá alguna de las
anteriores alternativas, o incluso alguna otra que se considere más
conveniente. Deberá, eso sí, justificarse el porqué de tal elección.



\subsection{Performance Requirements}
\label{sec:org94bc543}

The following are the performance requirements for the Cyber Planeta system:\\
\\
\textbf{System Load:}

\begin{itemize}
    \item \textbf{Number of terminals:} 50 concurrent terminals
    \item \textbf{Number of simultaneously connected users: }200 users
    \item \textbf{Transactions per second: }10 transactions per second
    \item \textbf{Response time:} 2 seconds maximum for each transaction
\end{itemize}
\\
\textbf{Data Requirements:}
\begin{itemize}
    \item \textbf{Usage frequency:} High usage frequency during tax and taxes declaration periods.
    \item \textbf{Access capabilities:} Simultaneous access to the database for multiple users.
    \item \textbf{Number of records:} It is expected to store a minimum of 10,000 tax declaration records and a maximum of 50,000 records
    \item \textbf{Database size:} The database is expected to be at least 100 GB in size and up to 500 GB
    \item \textbf{Read and write speed:} A read and write speed of at least 100 MB/s is required    
\end{itemize}
\textbf{Storage Requirements:}
\begin{itemize}
    \item \textbf {Storage space:} A minimum of 500 GB of storage space is required for the database and attachments.
    \item \textbf {Storage type:} Hard disk or SSD storage with fast access speed is required.
\end{itemize}

\textbf{Network Requirements:}
\begin{itemize}
     \item \textbf {Bandwidth:} At least 100 Mbps of bandwidth is required to ensure fast and secure communication.
     \item \textbf {Connectivity:}  Stable and secure connectivity is required through network protocols.
\end{itemize}
\textbf{Security Requirements:}
\begin{itemize}
    \item \textbf {Authentication:} Secure authentication is required for system users and administrators.
    \item \textbf {Authorization:} Secure authorization is required to access system data and functionalities.
    \item \textbf {Data integrity:} Data must be stored and transmitted securely and confidentially
\end{itemize}

It is important to note that these requirements may vary depending on the specific needs of the system and its users, so it is important to test and adjust the system to ensure optimal performance.



\subsection{Design Constraints}
\label{sec:org49fe900}

The following are the design constraints that may affect the development of the Cyber Planeta system:

\textbf{Standards Constraints}
\begin{itemize}
    \item Compliance with current tax regulations
    \item Compliance with cybersecurity information security standards
    \item Compliance with accessibility standards for people with disabilities
\end{itemize}

\textbf{Hardware Limitations}
\begin{itemize}
    \item The application must be compatible with the cybersecurity computer equipment, which has the following characteristics:
    \begin{itemize}
        \item \textbf{Processor:} Intel Core i5 or higher
        \item \textbf{RAM memory:} 8 GB or higher
        \item \textbf{Operating System:} Windows 10 or higher
        \item \textbf{Connectivity:} Wi-Fi or Ethernet
    \end{itemize}
    
    \item The application must be able to function in network environments with low connection speed
\end{itemize}

\textbf{Software Limitations}
\begin{itemize}
    \item The application must be compatible with the following web browsers:
    \begin{itemize}
        \item Google Chrome
    \end{itemize}
    
    \item The application must be able to work with different versions of the aforementioned web browsers.
    \item The application must be compatible with the following mobile operating systems:
    \begin{itemize}
        \item Android 8.0 or higher
    \end{itemize}
\end{itemize}
   
\textbf{Integration Constraints}    
\begin{itemize}
    \item The application must be able to integrate with the cybersecurity's existing tax management systems.
    \item The application must be able to integrate with the cybersecurity's electronic payment systems.
\end{itemize}

\textbf{Security Constraints}    
\begin{itemize}
    \item The application must comply with cybersecurity information security standards, including user authentication and authorization.
    \item The application must be able to encrypt sensitive and confidential data.
    \item The application must be able to detect and prevent common security attacks, such as SQL injections and cross-site scripting (XSS).
\end{itemize}

\textbf{Usability Constraints}    
\begin{itemize}
    \item The application must be easy to use and accessible to cybersecurity users.
    \item The application must be able to provide an intuitive and user-friendly experience.
    \item The application must be able to provide online help and support for users.
\end{itemize}

\subsection{System attributes}
\label{sec:orgd0babc0}

\textbf{Reliability}    
\begin{itemize}
    \item The system must be able to function properly for an extended period of time without errors or failures.
    \item The system must be able to recover quickly from a failure or error.
    \item The system must be able to provide a consistent and accurate response to user requests.
\end{itemize}

\textbf{Maintainability}    
\begin{itemize}
    \item The system must be easy to maintain and update.
    \item The system must be able to be modified or updated without affecting its operation.
    \item The system must be able to be scaled to adapt to changes in demand or user requirements.
\end{itemize}

\textbf{Portability}    
\begin{itemize}
    \item The system must be able to function in different environments and platforms.
    \item The system must be able to be executed on different operating systems and devices.
    \item The system must be able to be accessed from different locations and devices.
\end{itemize}

\textbf{Security}    
\begin{itemize}
    \item The system must be able to protect the confidentiality, integrity, and availability of data.
    \item The system must be able to authenticate and authorize users to access data and functionalities.
    \item The system must be able to detect and prevent common security attacks, such as SQL injections and cross-site scripting (XSS).
\end{itemize}

\textbf{Access Security}    
\begin{itemize}
    \item Authorized users will be those who have a valid account and have been successfully authenticated.
    \item Unauthorized users will not be able to access the data and functionalities of the system.
    \item Users will have different levels of access depending on their role and responsibilities.
\end{itemize}

\textbf{Security Mechanisms}    
\begin{itemize}
    \item \textbf{Authentication:}A username and password-based authentication system will be used.
    \item \textbf{Authorization:} A role and permission-based authorization system will be used.
    \item \textbf{Encryption:} A secure encryption algorithm will be used to protect data in transit and at rest.
    \item \textbf{Firewall:} A firewall will be used to protect the system from external attacks.
    \item \textbf{Monitoring:} The system will be continuously monitored to detect and prevent security attacks.
\end{itemize}

\textbf{Security Mechanisms}    
\begin{itemize}
    \item \textbf{Administrators:}will have full access to the system and will be able to perform any task.
    \item \textbf{Registered users: } will have access to the system's functionalities and will be able to perform specific tasks according to their role.
    \item \textbf{Guest users:} will have limited access to the system and will be able to perform specific tasks according to their role.
\end{itemize}

\subsection{"Other requirements}
\label{sec:org31d2978}

N/A

\newpage


\section{Appendix}
\label{sec:org75cea03}

Appendices can contain all kinds of information relevant to the ERS but that does not, strictly speaking, form part of the ERS. For example:

\begin{enumerate}
\item Data input/output formats, on screen or in listings.
\item Cost analysis results.
\item "Programming language restrictions.
\end{enumerate}
\end{document}