\documentclass[a4paper]{article}

\usepackage[english]{babel}
\usepackage[utf8]{inputenc}
\usepackage{amsmath}
\usepackage{graphicx}
\usepackage[colorinlistoftodos]{todonotes}

\title{Requirement's Interview Transcription}

\author{Code Creators}

\date{May 19, 2024}

\begin{document}
\maketitle


\section{What kind of tax processes do you perform?}

In response to the first question, I can say that the most common tax processes handled are VAT declarations, monthly declarations, semi-annual declarations, and withholding tax declarations.

\section{What steps do you follow to carry out each type of procedure? What is the customer service procedure?}

The steps for each procedure are different. As for the declarations, the most important thing is the RUC and the taxpayer's password. \\
Currently most of the information is electronic and practically everything is already on the internet. Invoicing is electronic, purchases are electronic, withholding is electronic, and so what we do is check all the electronic documentation that is in the system to have a clear idea of how much has been bought, how much has been sold, how much has been withheld. Practically 90 \% of the information is there.\\
 The only thing that is requested are the sales notes, which are not electronic information but physical information, which are the only thing that is requested when the declarations are made in case they have this type of document to complete the tax reconciliation, depending on that there will be values to pay or they will remain with tax credit, that is, VAT in favor.\\
 Income taxes are annual declarations and the annexes each have their own procedure, they are variable requirements, for example, in relation to the address of the dependency, which are those of the employees, it is necessary to have access to the IESS portal to verify the data of each employee and the salaries they have had, the contributions that have been made through their salary as well, to have a clear understanding of the issue of the thirteenth and fourteenth months, family burdens, the projection of personal expenses to make the complete information of each taxpayer, these are annexes that are done annually, at the end of the year, because it is required for those who have employees affiliated to the IESS. The shareholder annex is also an annex that is done annually, as the word says, each shareholder is detailed to see with what percentage of participation they develop within the company. Each process has its own procedure, they are quite different.
 
\section{What are the main data/documents that you need from a client to carry out the service?}

I have already explained four of the data, the documentation we request is only related to sales notes because they are through a booklet. The rest is already electronic, we no longer request this because it is in the SRI database, so what we request is the access key and the RUC, the rest we already check, we assemble and add the sales notes.

\section{At the end of the process, what kind of information or documentation is delivered to the client?}

At the end of the declarations, mainly VAT, withholding tax, or income tax declarations, the declaration receipt and the declaration detail are delivered. Although this is only for taxpayers to have as a backup, because in reality all the information is left on the internet and at any time they can reprint the data or documentation they need so that they can also keep track of their financial situation for the annual closing.

\section{Are your clients only individuals or also companies?}

Most of my clients are individuals. Companies, also some of those that are not obliged to keep accounting, because there are companies that are not obliged to keep accounting that are in a special regime. And I have also carried out tax processes for companies that are obliged to keep accounting, as long as they provide me with the two keys, because that information is sent with two keys: the company's key and the company's accountant's key.

\section{How many procedures do you carry out in a day?}

Well, the number of clients varies. On a busy day, let's say, during the months of highest demand, we can even run out of hours, reaching up to 70, 80, or 90 client declarations per day. There are also days, usually Saturdays when I sometimes work, that the number drops to 15-20 declarations. Remember that there are not only declarations, but also VAT refund procedures, and we have to check the taxpayers' invoices, which is usually done for a few months.

\section{Is there any specific functionality that you would like to add to the program for the utility of your business?}
As I mentioned, most of the information is already on the SRI website (forms, databases, etc.). What I would like is a type of agenda that can classify users with a declaration reminder since it is handled by the ninth digit of the RUC, for example, those with the number 1 have to declare on the 10th of each month, those with the number 2 on the 12th of each month, and so on every two days the digits increase until the last one which is the 0 that has to declare on the 28th of each month. So if it could be grouped and searched by the ninth digit of the RUC, and see which taxpayers have to declare on a certain day, it would be fabulous.\\
Also, if a functionality could be added to store the accounting issues of each client (sales notes and invoices) to be able to do an annual review at the end of the year. And another one for me to be able to calculate the fees for my service, including cost per declaration, type of declaration, assign additional costs according to the number of invoices, in such a way that it can facilitate the collection process for the process/processes that a client needs.





\end{document}